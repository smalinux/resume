%% The MIT License (MIT)
%%
%% Copyright (c) 2015 Daniil Belyakov
%%
%% Permission is hereby granted, free of charge, to any person obtaining a copy
%% of this software and associated documentation files (the "Software"), to deal
%% in the Software without restriction, including without limitation the rights
%% to use, copy, modify, merge, publish, distribute, sublicense, and/or sell
%% copies of the Software, and to permit persons to whom the Software is
%% furnished to do so, subject to the following conditions:
%%
%% The above copyright notice and this permission notice shall be included in all
%% copies or substantial portions of the Software.
%%
%% THE SOFTWARE IS PROVIDED "AS IS", WITHOUT WARRANTY OF ANY KIND, EXPRESS OR
%% IMPLIED, INCLUDING BUT NOT LIMITED TO THE WARRANTIES OF MERCHANTABILITY,
%% FITNESS FOR A PARTICULAR PURPOSE AND NONINFRINGEMENT. IN NO EVENT SHALL THE
%% AUTHORS OR COPYRIGHT HOLDERS BE LIABLE FOR ANY CLAIM, DAMAGES OR OTHER
%% LIABILITY, WHETHER IN AN ACTION OF CONTRACT, TORT OR OTHERWISE, ARISING FROM,
%% OUT OF OR IN CONNECTION WITH THE SOFTWARE OR THE USE OR OTHER DEALINGS IN THE
%% SOFTWARE.

% The font could be set to Windows-specific Calibri by using the 'calibri' option
\documentclass[]{Sohaib-Mohamed}

% For mathematical symbols
\usepackage{amsmath}
% href
\usepackage[hidelinks]{hyperref}
\usepackage{amssymb}


% Set applicant's personal data for header
\name{Sohaib Mohamed}
\address{
   Software Engineer, \linebreak
   Linux System Programmer \linebreak
   %2 years experience \linebreak
   Vienna, Austria
}
\contacts{
   \href{mailto:sohaib.amhmd@gmail.com}{sohaib.amhmd@gmail.com} \linebreak
   \textbf{LinkedIn: }\href{https://linkedin.com/in/smalinux}{smalinux} \linebreak
   \textbf{GitHub:} \href{https://github.com/smalinux}{smalinux} \linebreak
   \textbf{WhatsApp: }\href{https://wa.me/4368110477349}{\textit{4368110477349}}
}

\begin{document}

% Print the header
\makeheader


%%%%%%%%%%%%%%%%%%%%%%%%%%%%%%%%%%%%%%%%%
%%% Experience
%%%%%%%%%%%%%%%%%%%%%%%%%%%%%%%%%%%%%%%%%
\begin{cvsection}{Experience}

   %%%%%%%%
   \begin{cvsubsection}{Software Engineer}{LOYTEC electronics GmbH}{Jan 2024 - Present}
      \begin{itemize}
         \item Maintenance and expansion of an existing Buildroot/Linux environment.
         \item Development of new functions and modules at the operating system level.
         \item Software-side support for the hardware department in evaluating and commissioning new hardware.
      \end{itemize}
      %\textbf{Skills}: x, y, z
   \end{cvsubsection}

   %%%%%%%%
   \begin{cvsubsection}{Software Engineer, Intern}{Google Summer of Code}{June 2022 - Oct 2022}
      \textbf{Link:} \href{https://gist.github.com/smalinux/e869b376b5c77cacdcda4cb14f027632}{\textit{GitHub}} | \textbf{Read More:} \href{https://gist.github.com/smalinux/2e9c5537fdac65501a655280352c9c15#google-summer-of-code-2022}{\textit{Portfolio}}
      \begin{itemize}
         \item Added \approx10,000 lines of production-ready code written in \textbf{C language} and \textbf{bash script}.
         \item Ported 12 \textbf{eBPF} performance tools to the PCP framework.
         \item Built a framework to extend Htop columns by 75\%, allowing for seamless integration of various performance metrics by implementing a feature called \href{https://github.com/htop-dev/htop/pull/1102}{\textit{Dynamic Screens}}.
      \end{itemize}
      %\textbf{Skills}: x, y, z
   \end{cvsubsection}

   %%%%%%%%
   \begin{cvsubsection}{Open-Source Contributor}{Linux Kernel | Htop | PCP}{Feb 2021 -- Present}
      \textbf{Links:} \href{https://git.kernel.org/pub/scm/linux/kernel/git/next/linux-next.git/log/?qt=grep&q=sohaib}{\textit{Linux Kernel}} | \href{https://github.com/htop-dev/htop/commits?author=smalinux}{\textit{Htop}} | \href{https://github.com/performancecopilot/pcp/commits?author=smalinux}{\textit{PCP}}
      \begin{itemize}
         \item Contributed to the Linux kernel development by submitting patches to \textbf{the kernel} to resolve memory leaks.
         \item Designed several successful new features requiring in-depth \textbf{research} on unfamiliar topics before coding.
         \item Implementing the \href{https://github.com/htop-dev/htop/pull/707}{\textit{Dynamic Columns}} and \href{https://github.com/htop-dev/htop/pull/669}{\textit{Dynamic Meters}} features resulted in an 80\% improvement in overall \href{https://man.archlinux.org/man/pcp-htop.1.en}{\textit{pcp-htop}} system efficiency.
      \end{itemize}
      %\textbf{Skills}: x, y, z
   \end{cvsubsection}
\end{cvsection}


%%%%%%%%%%%%%%%%%%%%%%%%%%%%%%%%%%%%%%%%%
%%% Projects
%%%%%%%%%%%%%%%%%%%%%%%%%%%%%%%%%%%%%%%%%
\begin{cvsection}{Projects}

   %%%%%%%%
   \begin{cvsubsection}{Linux kernel modules}{}{Nov 2021 -- Jun 2021}
      \textbf{Link:} \href{https://github.com/smalinux/linux-kernel-modules-lab}{\textit{GitHub}} | \textbf{Read More:} \href{https://gist.github.com/smalinux/2e9c5537fdac65501a655280352c9c15#linux-kernel-modules}{\textit{Portfolio}}
      \begin{itemize}
         \item 80+ Linux kernel modules written while working through the \href{https://github.com/cirosantilli/linux-kernel-module-cheat}{\textit{LKMC}} repository and the \textit{LDD3} book.
         \item Learning how to use kernel APIs helped me acquire practical \textbf{operating system} and \textbf{low-level} knowledge.
      \end{itemize}
      \textbf{Skills}: linux kernel API and subsystems; System programming; C; GNU Make
   \end{cvsubsection}

   %%%%%%%%
   \begin{cvsubsection}{SimOS}{}{Mar 2020 - Jan 2021}
      \textbf{Link:} \href{https://github.com/smalinux/simOS}{\textit{GitHub}} | \textbf{Read More:} \href{https://gist.github.com/smalinux/2e9c5537fdac65501a655280352c9c15#simos}{\textit{Portfolio}}
      \begin{itemize}
         \item Tiny operating system was written using pure \textbf{assembly} code only while working through osdev.org.
         \item Gained a deep knowledge of computer system architecture and design.
      \end{itemize}
      \textbf{Skills}: assembly language; system architecture; kernel development; low-level programming
   \end{cvsubsection}
\end{cvsection}


%%%%%%%%%%%%%%%%%%%%%%%%%%%%%%%%%%%%%%%%%
%%% Education
%%%%%%%%%%%%%%%%%%%%%%%%%%%%%%%%%%%%%%%%%
\begin{cvsection}{Education}
   \begin{cvsubsection}{Giza, EG}{Misr University}{Oct 2019 -- Aug 2022}
      B.S. in Computer Science. Misr University for Science and Technology -- CGPA 2.75 / 4 -- major GPA: 3 / 4
   \end{cvsubsection}
\end{cvsection}


%%%%%%%%%%%%%%%%%%%%%%%%%%%%%%%%%%%%%%%%%
%%% Awards
%%%%%%%%%%%%%%%%%%%%%%%%%%%%%%%%%%%%%%%%%
\begin{cvsection}{Awards}
   \begin{cvsubsection}{}{}{}
      \begin{itemize}
         \item \textbf{Added as a Maintainer for \href{https://github.com/performancecopilot}{\textit{Performance Co-Pilot}} Organization:} because of the proven track record of past open-source contributions: \href{https://github.com/performancecopilot}{\textit{https://github.com/performancecopilot}}.
         \item \textbf{Recognized as a Top ``Tested-by" Contributor to Linux Kernel 5.16:} \href{https://lwn.net/Articles/880699/}{\textit{https://lwn.net/Articles/880699}}.
      \end{itemize}
   \end{cvsubsection}
\end{cvsection}


%%%%%%%%%%%%%%%%%%%%%%%%%%%%%%%%%%%%%%%%%
%%% Skills
%%%%%%%%%%%%%%%%%%%%%%%%%%%%%%%%%%%%%%%%%
\begin{cvsection}{Skills}
   \begin{cvsubsection}{}{}{}
      %%%%%%%% max: 211 char
      C language (Advanced); Python; Linux; Docker; Git; Linux kernel; System programming;
   \end{cvsubsection}
\end{cvsection}

\end{document}
